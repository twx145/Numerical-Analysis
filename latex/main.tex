% !TEX program = xelatex
% !TEX encoding = UTF-8

\documentclass[12pt, a4paper, dotinlabels]{article}

% ======================================================================
% 1. 宏包设置 (Preamble)
% ======================================================================

% --- 页面与字体设置 ---
\usepackage{geometry}
\geometry{a4paper, left=2.5cm, right=2.5cm, top=2.5cm, bottom=2.5cm} % 设置页边距

\usepackage{fontspec} % 允许设置字体
\setmainfont{Times New Roman} % 设置英文字体
\usepackage[UTF8]{ctex} % 中文支持宏包,自动设置中文字体

% --- 设置全局中文字体方案 ---
\setCJKmainfont{SimSun}  % 设置中文主字体为宋体
\setCJKsansfont{SimHei}  % 设置中文无衬线字体为黑体

% --- 数学公式相关 ---
\usepackage{amsmath} % AMS数学宏包
\usepackage{amssymb} % AMS数学符号宏包

% --- 图形与表格相关 ---
\usepackage{graphicx} % 插入图片
\usepackage{caption} % 自定义图表标题
\captionsetup{labelsep=space, justification=centering, font=small} % 图表标题设置
\usepackage{subcaption} % 子图
\usepackage{booktabs} % 三线表
\usepackage{longtable} % 跨页表格
\usepackage{multirow} % 合并表格单元格

% --- 参考文献设置 (使用 biblatex 和 biber) ---
\usepackage[
backend=biber,       % 使用biber作为后端
style=gb7714-2015,   % 符合国标 GB/T 7714-2015 的参考文献样式
sorting=none         % 按引用顺序排序
]{biblatex}
\addbibresource{references.bib} % 关联参考文献数据库文件

% --- 其他常用宏包 ---
\usepackage{hyperref} % 创建超链接
\hypersetup{
	colorlinks=true,
	linkcolor=black,
	filecolor=magenta,
	urlcolor=cyan,
	pdftitle={数值分析方程的迭代解法研究报告},
	pdfpagemode=FullScreen,
}

\usepackage{fancyhdr} % 设置页眉页脚
% 样式1:'fancy' - 用于正文,带有页眉和横线
\pagestyle{fancy}
\fancyhf{}
\fancyhead[C]{\leftmark}
\fancyfoot[C]{\thepage}
\renewcommand{\headrulewidth}{0.4pt}
\renewcommand{\footrulewidth}{0.4pt}
% 样式2:'plainfancy' - 用于摘要、目录等前置部分,没有页眉和横线
\fancypagestyle{plainfancy}{
	\fancyhf{} % 清空页眉页脚
	\fancyfoot[C]{\thepage} % 只在页脚中间显示页码
	\renewcommand{\headrulewidth}{0pt} % 页眉线宽度为0
	\renewcommand{\footrulewidth}{0pt} % 页脚线宽度为0
}

\usepackage{titlesec} % 自定义章节标题格式
\titleformat{\section}{\Large\bfseries\sffamily}{\thesection}{1em}{}
\titleformat{\subsection}{\large\bfseries\sffamily}{\thesubsection}{1em}{}
\titleformat{\subsubsection}{\normalsize\bfseries}{\thesubsubsection}{1em}{}

\usepackage{tocloft}
\renewcommand{\cftsecleader}{\cftdotfill{\cftdotsep}} % for \section
\renewcommand{\cftsubsecleader}{\cftdotfill{\cftdotsep}} % for \subsection
\renewcommand{\cftsubsubsecleader}{\cftdotfill{\cftdotsep}} % for \subsubsection
\renewcommand{\cftdotsep}{2.5}
\setlength{\cftbeforesecskip}{1em}  % 调整 \section 条目之前的距离
\setlength{\cftbeforesubsecskip}{0.5em} % 调整 \subsection 条目之前的距离

\usepackage{setspace}
\setlength{\cftbeforesecskip}{0pt}
\setlength{\cftbeforesubsecskip}{0pt}

% 自定义 abstract 环境
\renewenvironment{abstract}
{\small
	\begin{center}
		{\Large\bfseries\sffamily\abstractname\vspace{1em}\vspace{0pt}}
	\end{center}
	\quotation}
{\endquotation}
\renewcommand{\contentsname}{{\sffamily\bfseries 目录}}


% ======================================================================
% 2. 文档信息
% ======================================================================

\title{数值分析方程的迭代解法研究报告}
\author{佟文轩}
\date{\today}


% ======================================================================
% 3. 文档正文开始
% ======================================================================

\begin{document}
	
	% --- 详细封面页 ---
	\begin{titlepage}
		\begin{center}
			
			% 顶部可以添加校徽或学校名称
			% \includegraphics[width=4cm]{logo.png} % 如果有校徽图片
%			\vspace*{1.5cm} % 顶部留白
%			{\LARGE \bfseries 某某大学本科生毕业论文\par} % 学校和论文类型
%			\vspace{0.5cm}
%			\rule{\textwidth}{1pt} % 一条分割线
			
			\vspace*{3.5cm}
			
			% --- 论文题目 ---
			{\Huge \bfseries 数值分析方程的迭代解法研究报告\par}
			
			\vspace{8cm}
			
			% --- 作者、学号等详细信息 ---
			\begin{tabular}{l@{\hspace{1em}}l} % l代表左对齐,@{}中间是列间距
				姓    \quad 名: & \Large 佟文轩 \\
				\addlinespace[1em] % 增加行间距
				学    \quad 号: & \Large 1120240934 \\
				\addlinespace[1em]
				专    \quad 业: & \Large 计算机科学与技术 \\
				\addlinespace[1em]
				指导老师: & \Large 孙新 \\
			\end{tabular}
			
			\vfill % 将下面的内容推到底部
			
			% --- 日期 ---
			{\large \today\par}
			
		\end{center}
	\end{titlepage}
	
	% --- 切换到正文页码和样式 ---
	\newpage
	\pagenumbering{roman}
	\pagestyle{plainfancy}
	
	% --- 摘要与关键词 ---
	\begin{abstract}
		研究目的、方法、结果和结论。
		
		在数值分析的第二章课程中我们学习了非线性方程的迭代解法,这一类方法是为计算机量身定制的,不同于以往我们在学习中接触到的解析式解法,因此,本报告的目的是在计算机中实际应用上述迭代方法,以加深自己对迭代解法的理解。
		
		本报告选择使用JAVA语言来进行上机实现,原因有三:其一为本学期新增JAVA选修课可以借此机会增加对JAVA语言的熟练度;其二JAVA还提供较为方便的javafx方便将我的程序更好的可视化;其三JAVA可以将我的程序方便的转化为软件,方便他人一键安装使用。
		
		在上机期间我手动编写了有关迭代方法————牛顿法、艾特肯迭代法、单点弦截法等9种方法,其余部分借助AI完成相关内容的编写。该软件
		
		\vspace{1em} % 增加一点垂直间距
		\noindent\textbf{关键词:} 方程迭代解法;牛顿迭代法;可视化软件
	\end{abstract}
	
	\newpage
	
	% --- 目录 ---
	\begin{center}
		\begin{spacing}{1.3}
			\tableofcontents % 目录生成命令
		\end{spacing}
	\end{center}
	
	\newpage
	% --- 设置正文部分的页码为阿拉伯数字, 并从1开始 ---
	\pagenumbering{arabic} % 此命令会自动将页码重置为1
	\pagestyle{fancy}
	
	% ======================================================================
	% 4. 论文正文各章节
	% ======================================================================
	
	\section{引言}
	\label{sec:intro}
	
	这是引言部分。您可以在这里介绍研究背景、意义以及论文的主要工作和结构。
	
	在LaTeX中,您可以使用 `\cite{}` 命令来引用参考文献。例如,Knuth的经典著作《The TeXbook》 \cite{knuth1984} 是一本非常好的参考书。您也可以同时引用多篇文献,例如 \cite{einstein1905, dirac1928}。
	
	\section{迭代解法的原理}
	\label{sec:related}
	
	在这一章中,您可以回顾与您的研究相关的先前工作。
	
	\subsection{理论基础}
	\label{ssec:theory}
	介绍相关的理论和公式。例如,爱因斯坦的质能方程:
	\begin{equation}
		E = mc^2
		\label{eq:emc2}
	\end{equation}
	我们可以通过 `\ref{eq:emc2}` 来引用公式 \eqref{eq:emc2}。
	
	\subsection{技术现状}
	\label{ssec:state-of-art}
	介绍当前的技术发展水平。
	
	\section{java软件实现}
	\label{sec:method}
	
	详细描述您所采用的研究方法。
	
	\subsection{图表示例}
	下面是一个插入图片的示例(如图\ref{fig:example}所示)。
	
	\begin{figure}[htbp]
		\centering
		\includegraphics[width=0.5\textwidth]{example-image-a} % 确保您有这个图片文件,或者使用您自己的图片
		\caption{这是一个示例图片}
		\label{fig:example}
	\end{figure}
	
	\subsection{表示例}
	下面是一个使用 `booktabs` 宏包创建三线表的示例(如表\ref{tab:example}所示)。
	
	\begin{table}[htbp]
		\centering
		\caption{这是一个示例表格}
		\label{tab:example}
		\begin{tabular}{lcr}
			\toprule
			\textbf{姓名} & \textbf{专业} & \textbf{年龄} \\
			\midrule
			张三 & 计算机科学 & 22 \\
			李四 & 物理学 & 23 \\
			王五 & 化学 & 21 \\
			\bottomrule
		\end{tabular}
	\end{table}
	
	
	\section{迭代方法对比}
	\label{sec:results}
	
	展示您的实验设置、过程和获得的结果。
	
	
	\section{总结}
	\label{sec:conclusion}
	
	总结您的工作,并指出未来的研究方向。
	
	
	
	% ======================================================================
	% 5. 参考文献
	% ======================================================================
	\newpage
	% --- 设置正文部分的页码为阿拉伯数字, 并从1开始 ---
	\pagenumbering{arabic}
	\begin{center}
		{\Large\bfseries\sffamily 参考文献}
	\end{center}
	\printbibliography[heading=none]
	\markboth{参考文献}{}
	
	
	
	% ======================================================================
	% 6. 致谢 (可选)
	% ======================================================================
	\newpage
	\begin{center}
		{\Large\bfseries\sffamily 致谢}
	\end{center}
	
	\addcontentsline{toc}{section}{致谢} % 将致谢添加到目录中
	\markboth{致谢}{}
	
	衷心感谢孙新老师在《数值分析》课程中的指导与帮助!报告不足之处,恳请指正。
	
	
	% ======================================================================
	% 7. 附录 (可选)
	% ======================================================================
	\newpage
	\appendix
	\begin{center}
		{\Large\bfseries\sffamily 附录}
	\end{center}
	\addcontentsline{toc}{section}{附录}
	\markboth{附录}{}
	
	
	\section{高斯-赛德尔迭代法MATLAB代码}
	\label{app:code}
	
	这里是附录A的内容...
	
	
	\section{实验原始数据记录}
	\label{app:data}
	
	这里是附录B的内容...
	
	这里是附录A的内容。
	
\end{document}